%There are two major modes of typesetting math in LaTeX. one is embedding the math directly into your text by encapsulating your formula in dollar signs and the other is using a predefined math environment. You can follow along and try the code in the sandbox below. I also prepared a quick reference of math symbols.

%Using inline math - embed formulas in your text
%To make use of the inline math feature, simply write your text and if you need to typeset a single math symbol or formula, surround it with dollar signs

\iffalse
\documentclass{article}
\begin{document}

This formula $f(x) = x^2$ is an example.

\end{document}
\fi

%The equation and align environment:

%The most useful math envorinments are the equation environment for typesetting single equations and the align environment for multiple equations and automatic alignment:
\iffalse
\documentclass{article}

\usepackage{amsmath}

\begin{document}

\begin{equation*}
  1 + 2 = 3 
\end{equation*}

\begin{equation*}
  1 = 3 - 2
\end{equation*}

\begin{align*}
  1 + 2 &= 3\\
  1 &= 3 - 2
\end{align*}

\end{document}
\fi 
%The align environment will align the equations at ampersand &. (That is to say, the thing after & will be aligned, in this case "="). Single equations have to be seperated by a linebreak \\. There is no alignment when using the simple equation environment. Furthermore it is not even possible to enter two equations in that environment, it will result in a compilation error. The asterisk (e.g. equation*) only indicates, that I don't want the equations to be numbered.


%Fractions and more:

%LaTeX is capable of displaying any mathematical notation. It's possible to typeset integrals, fractions and more. Every command has a specific syntax to use. I will demonstrate some of the most common LaTeX math features:
\documentclass{article}

\usepackage{amsmath}

\begin{document}

\begin{align*}
  f(x) &= x^2\\
  g(x) &= \frac{1}{x}\\
  F(x) &= \int^a_b \frac{1}{3}x^3
\end{align*}

\begin{align*}
  &f(x) = x^2\\
  &g(x) = \frac{1}{x}\\
  &F(x) = \int^a_b \frac{1}{3}x^3 \\
  &\frac{1}{\sqrt{x}}
\end{align*}

\begin{equation}
     1 = 3 - 2 %numbered
\end{equation}

\begin{equation*}
     1 = 3 - 2 %not numbered
\end{equation*} 

%Matrices:
%need double $
$$ 
\begin{matrix}
1 & 0\\
0 & 1
\end{matrix}
$$
%Furthermore it's possible to display matrices in LaTeX. There is a special matrix environment for this purpose, please keep in mind that the matrices only work within math environments(package) as described above:

%Brackets in math mode - Scaling:
%To surround the matrix by brackets, it's necessary to use special statements, because the plain [ ] symbols do not scale as the matrix grows. The following code will result in wrong brackets:
$$
[
\begin{matrix}
1 & 0\\
0 & 1
\end{matrix}
]
$$




%To scale them up, we must use the following code:
$$
\left[
\begin{matrix}
1 & 0\\
0 & 1
\end{matrix}
\right]
$$

%or use a bmatrix
$$
\begin{bmatrix}
1 & 0\\
0 & 1
\end{bmatrix}
$$

%This does also work for parentheses and braces and is not limited to matrices. It can be used to scale for fractions and other expressions as well:

%not use $$, ask you to enter math mode.
% basically enters math mode
$$
\left(\frac{1}{\sqrt{x}}\right)
$$
\end{document}

%Summary:
%LaTeX is a powerful tool to typeset math
%Embed formulas in your text by surrounding them with dollar signs $
%The equation environment is used to typeset one formula
%The align environment will align formulas at the ampersand & symbol
%Single formulas must be seperated with two backslashes \\
%Use the matrix environment to typeset matrices
%Scale parentheses with \left( \right) automatically
%All mathematical expressions have a unique command with unique syntax
%Notable examples are: \int^a_b for integral symbol; \frac{u}{v} for fractions; \sqrt{x} for square roots
%Characters for the greek alphabet and other mathematical symbols such as \lambda

%The more complex the expression, the more error prone this is, it's important to take care of opening and closing the braces {}. It can take a long time to debug such errors. The Lyx program offers a great formula editor, which can ease this work a bit. Personally, I write all code by hand though, since it's faster than messing around with the formula editor.

