%LaTeX offers a lot of functions by default, but in some situations it can become in handy to use so called packages. To import a package in LaTeX, you simply add the \usepackage directive to the preamble of your document:

%\documentclass{article}

%\usepackage{PACKAGENAME}

%\begin{document}


%Installation of Packages.
%When using Linux or Mac, most packages will already be installed by default and it is usually not necessary to install them. In case of Ubuntu installing texlive-full from the package manager would provide all packages available. The MiKTeX bundle in Windows, will download the package if you include it to your document.

%Use of Packages.
%There are countless packages, all for different purposes in my tutorials I will explain some of the most useful. To typeset math, LaTeX offers (among others) an environment called equation. Everything inside this environment will be printed in math mode, a special typesetting environment for math. LaTeX also takes care of equation numbers for us:

%Normally a equation number is included:

\iffalse
\documentclass{article}

\begin{document}

\begin{equation}
  f(x) = x^2
\end{equation}

\end{document}
\fi
% f = (x^2) (1)

%Include a package.
%The automatic numbering is a useful feature, but sometimes it's necessary to remove them for auxiliary calculations. LaTeX doesn't allow this by default, now we want to include a package that does:

\documentclass{article}

\usepackage{amsmath}

\begin{document}

\begin{equation*}
  f(x) = x^2
\end{equation*}

\end{document}
%Now we get the same output as before, only the equation number is removed: f(x)=x^2


%Summary:
%Packages add new functions to LaTeX
%All packages must be included in the preamble
%Packages add features such as support for pictures, links and bibliography

