%This second example differs slightly from the first one, since this command involves a \begin and \end statement. 

%In fact this is not a command, but defines an environment. An environment is simply an area of your document where certain typesetting rules apply. 

%It is possible (and usually necessary) to have multiple environments in a document, but it is imperative (important) the document environment is the topmost environment.

\iffalse
% Valid:

% Valid:

\begin{document}
  \begin{environment1}
    \begin{environment2}
    \end{environment2}
  \end{environment1}
\end{document}

%Invalid:

%\begin{document}
%  \begin{environment1}
%    \begin{environment2}
%  \end{environment1}
%    \end{environment2}
%\end{document}

% Invalid:

%\begin{document}
%  \begin{environment1}
%\end{document}
%  \end{environment1}

% Also invalid:

%\begin{environment}
%  \begin{document}
%  \end{document}
%\end{environment}
%\fi

%There are numerous choices for environments and you will most likely need them as soon as you introduce large parts of mathematics or figures to your document. While it is possible to define your own environments, it is very likely that the environment you desire already exists. LaTeX already comes with a few predefined environments and even more come in so called packages, which are subject to another lesson later on.

%Obviously the statements \title, \date and \author are not within the the document environment, so they will not directly show up in our document. The area before our main document is called preamble.In this specific example we use it to set up the values for the \maketitle command for later use in our document.

%This command will automagically create a titlepage for us. The \newpage command speaks for itself.

%We can spot a page number at the bottom of our title page. What if we decide, that actually, we don't want to have that page number showing up there. We can remove it, by telling LaTeX to hide the page number for our first page. This can be done by adding the \pagenumbering{gobble} command and then changing it back to \pagenumbering{arabic} on the next page numbers like so:


%
\documentclass{article}

\title{My first document}
\date{2017-01-08}
\author{Alex Liu}

\begin{document}
  \pagenumbering{gobble} %Hide the page number.
  \maketitle
  \newpage
  \pagenumbering{arabic} %hanging it back to

  Hello World!
\end{document}


%Summary:
%%
%A document has a preamble and document part
%The document environment must be defined
%Commands beginning with a backslash \, environments have a begin and end tag
%Useful settings for pagenumbering:
%gobble - no numbers
%arabic - arabic numbers
%roman - roman numbers
%